\documentclass{article}
\usepackage[scale=0.85, footskip=0.8cm]{geometry} % Reduce document margins
\setlength{\parindent}{0pt}%
\addtolength{\parskip}{\baselineskip}

\usepackage{cite}
\usepackage{graphicx}
\usepackage{amsmath, xparse}
\usepackage{url}

\title{Using LaTeX in RStudio}
\author{Christine Chai}
\date{\today}

\begin{document}

\maketitle

\section{Intro}

There are multiple ways to compile a LaTeX document in RStudio.

\subsection{Compile within RStudio IDE}

In RStudio, it is possible to use the ``Compile PDF'' button to compile a \texttt{.tex} file. As long as you create or open a \texttt{.tex} file in RStudio, this ``Compile PDF'' button will automatically appear below your file tab. The \texttt{.tex} file will also be syntax-highlighted, making the code easier to read and debug.

However, this method generates many unwanted auxiliary files, including \texttt{.bbl}, \texttt{.log}, and \texttt{.synctex.gz}. We have to set these in \texttt{.gitignore}, and/or manually remove these intermediate files. Be careful not to remove the final PDF you created.

How to change the RStudio settings to automatically remove these auxiliary files:

\begin{enumerate}
\item Tools $\rightarrow$ Global Options $\rightarrow$ Sweave
\item Open: LaTeX Editing and Compilation
\item Check the box: Use tinytex when compiling .tex files
\item Check the box: Clean auxiliary output after compile
\end{enumerate}


\subsection{Compile using R Console}

Alternatively, you may also use the R Console to compile a \texttt{.tex} file into a PDF. In the $\mathsf{R}$ package \texttt{tools}, just use the \texttt{texi2pdf} (pronounced as tex-i2-pdf) command and set \texttt{clean=TRUE}. In this way, the auxiliary files are automatically removed. We keep only the \texttt{.tex} and PDF in the system.

\begin{verbatim}
library(tools)
texi2pdf("Test-R-Compile-PDF.tex", clean=TRUE)
\end{verbatim}

\section{Math Equations}

\(\mathbf{\emptyset_0} =\begin{bmatrix}0,1,0\end{bmatrix}\)

This is a $2 \times 2$ matrix: $\begin{bmatrix} x_{11} & x_{12}\\ x_{21} & x_{22} \end{bmatrix}$

\section{Discussion}

It is recommended to put the references \texttt{.bib} file in the same directory as your \texttt{.tex} file. But since RStudio is an IDE, you may also put the references \texttt{.bib} file in a separate folder and specify the path to it. It is preferrable to specify a relative path from the source directory, rather than an absolute path.

Although RStudio regards \texttt{.bib} files as regular text files by default, you may change the typesetting from ``Text file'' to ``TeX'' to turn on syntax highlighting. The typesetting button is located at the lower-right of your source code window in RStudio.

Test citation~\cite{chai-2023-rss-climate}

Second citation~\cite{wilkinson2013grammar}


\bibliography{Test_Folder/Test-R-Compile-Ref}
\bibliographystyle{plain}


\end{document}